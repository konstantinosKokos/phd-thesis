\chapter{Proof Extraction}
\label{chapter:chapter_3}

\chapabstract{\textit{What good is a program that only lives on paper?}}

With our theorycrafting over, we have in our hands an uninstantiated descriptive model of syntactic and semantic composition, promising to capture dependency relations while keeping both its feet set firmly in type theory.
Unfortunately, it is well attested by now that ``all models are wrong...''~\cite{doi:10.1080/01621459.1976.10480949}.
Any promise of theoretical universality, cognitive plausibility, linguistic intrinsicness or what have you, would require some degree of handwaving and conjencturing that I am not comfortable with.
What is undisputable, however, is the nobility of our goals and the purity of our tools: the \textit{omniversality} of the $\lambda$ calculus~\cite{wadler2015propositions} asserts that our modeling approach is not some ad hoc machinery designed to tackle a highly localized problem, but the one and only programming language ever worth writing. 
Beyond purpose and methodology, and having \textit{a priori} given up any ideation of truth, the only measure of success for our model is that of its utility (``...but some are useful'', ~\cite{doi:10.1080/01621459.1976.10480949}).
This sets up a new research imperative: we must prove the model useful!

``Oof, that's a tricky one'', you might say, and you wouldn't be wrong.
Thankfully, there are two tried and tested ways to proceed.
The first is the way of the scholar, which requires a rare combination of high intellectual capacity, strong persuasive skills and a pinch of luck.
It starts off with some profound abstract thinking, gradually overtaken by agressive campaigning: bashing the competition at workshops and conferences, making bold claims and generating traction at any chance given (optionally over social media platforms, for the modernists), eventually building a cult of personality, and finally resting on your laurels as the hype becomes self-sustainining; at long last, utility affirmed by popular approval.
This path, sometimes called the scientific method, is a rather involved and painstakingly slow process, a high stakes gamble that only starts yielding profits in the long run; as such, it greatly benefits from the nourishment of a stable work environment.
The alternative is the way of the engineer, a shorter term investment more befitting the modern paradigm of the mobile, adaptive, multi-purpose researcher.
It requires only the acquirable skills of endurance and hardheadedness, and offers a recipe that's easier to follow: simply swing at it until it cracks.
After a lot of obsessive iteration and self-correction (interchanged with the occasional feeling of despair and futility), utility will sooner or later be affirmed by cold, hard numbers.
We'll go for this one.

This choice has some methodological repercussions.
Under more tranquil circumstances, we'd wait for the theory to be disseminated, criticized, adapted, error-corrected and returned to sender, before finally moving on; it being a theory of language, this would entail a thorough qualitative analysis of several kinds of linguistic phenomena, coupled with a theoretical investigation of what it can or cannot adequately capture.
We are however in a compressed timeframe, forcing our hand into putting it straight to the test; the pragmatic approach is then to try and directly align it with real-world linguistic data at scale, and hope for the best.
The process, called \textit{proof extraction}, revolves around ``proving''  some source corpus of syntactically annotated sentences via the design and application of an algorithm tasked with translating the existing annotation format into derivations of the target grammar.
Proof extraction serves a ternary purpose.
One, it gives us access to an uncompromisingly realistic testbed upon which we can immediately inspect and iteratively finetune the specifics of the grammar logic.
Two, it fills in for a strict and impartial external critic in providing a quantitative evaluation regime -- at each point in time, we are able to measure the proportion of source analyses (and corresponding linguistic phenomena) the algorithm provides a (reasonable) output for.
And three, the end-yield of this process has merit of its own; as a derived dataset, it is first a building block necessary for populating the computational toolshed of the theory, and also a public resource for the world to with as they please.

\section{Preliminaries}

\subsection{The Dutch Language}
For our linguistic inquiries, the focus will be on Dutch.
Other than being the language I was contractually obliged to conduct this research on, Dutch is an interesting specimen, the idiosyncracies of which have in the past proven quite a topic of debate for others, and a source of headaches for myself.
I don't have any intention (or delusion of competence) to casually throw a detailed exposition of the Dutch grammar here, but a brief and superficial typological overview might help smooth the transition into the what is to come.
%Note that this is not going to spare us the headaches -- just deferring them to the near future, and giving us some time to prepare.

But first things first.
Dutch is a West Germanic language, spoken primarily in the Low Countries within Europe by some 25 million speakers.
Owing to the Netherlands' nasty colonial history, Dutch has left a noticeable mark on the global linguistic atlas: it has played a primary role in the evolution of Afrikaans (with which it is somewhat mutually intelligible) and Indonesian, to a far lesser extent, while native Dutch speakers can be found as far as South America and the Dutch Carribean region.
Demographics aside, the language is said to be one of the closest relatives of English and german, sharing many of their morphosyntactic characteristics -- we'll go through some of those together.
The abbrevations in the glosses to follow are industry standard; you can find their transcriptions in Table~\ref{table:gloss_abbreviations} of Appendix~\ref{sec:abbrevations}.

\subsubsection{The Noun Phrase}
\paragraph{Nouns}
Dutch nouns have three grammatical genders: the masculine, the feminine and the neuter, of which the first two (the nonneuter) are morphologically indistinguishable.
There's two numbers, the singular and the plural, the latter constructed by the affixation of \textex{-en} or \textex{-s} -- the choice of which depends on the noun (Gloss~\ref{gloss:genitive}).
Case markings are not overtly realized, except for some mostly frozen leftovers from the distant past -- their functionality has largely been replaced by word order constraints, with the indirect object (formerly in the dative) preceding the direct object (formerly in the accusative), and periphrastic constructions, with the preposition \texttr{aan}{to} used to indicate indirect objects, and \texttr{van}{of} to substitute the genitive (Gloss~\ref{gloss:genitive}).
A cute peculiarity of Dutch is the strikingly common use of a productive diminutive form, denoting either small size or an affectionate disposition -- it is accomplished by the affixation of \textex{-tje} or regionally \textex{-ke}, which turns the inflected noun neuter (Gloss~\ref{gloss:diminutive}).

\paragraph{Determiners}
Determiners precede the noun and match its gender and number.
There are two articles: the definite \textex{de}/\textex{het} (nonneuter/~neuter) with plural form \textex{de}, and the indefinite \textex{een}, which does not inflect for gender, and has no plural form (Glosses~\ref{gloss:simple_np}, \ref{gloss:adjective_definite_n}, \ref{gloss:superlative}, \ref{gloss:adjective_definite_nn}, \ref{gloss:diminutive}{} and \ref{gloss:adjp}).
Indefinite pronouns can be used to convey universal or existential quantification and negation, materializing as substitutes for determiners (e.g. \texttr{alles}{all}, \texttr{sommige}{some}, \texttr{geen}{no}, etc., Gloss~\ref{gloss:adjp}), or as stand-alones (e.g. \texttr{iets}{something}, \texttr{niets}{nothing}, etc.).
Possessive and demonstrative pronouns can also enact determiners, both uninflected (except for the first person plural \textex{ons} which inflects like an indefinite adjective, see next paragraph).
Demonstrative pronouns come in two flavours: the proximal \textex{deze/dit} (plural \textex{deze}) and the distal \textex{die/dat} (plural \textex{die}).
The last two do double duty as relative pronouns -- more on that later.

\paragraph{Adjectives}
Adjectives used as nominal modifiers find their place between the determiner and the head noun.
They appear inflected with an \textex{-e} affix in all cases except for their idefinite use with a neuter noun (Glosses~\ref{gloss:adjective_definite_n} and \ref{gloss:adjective_definite_nn}).
The same affix applies to nominalized adjectives, which can allow the ommission of a contextually implied noun, or be used independently as an abstract concept or a quantifying property.
An alternative inflection with a \textex{-s} affix marks the partitive use, when the adjective modifies an indefinite pronoun (Gloss~\ref{gloss:partitive}).
The language provides acccess to a comparative and a superlative form, via the affixes \textex{-er} and \textex{-ste} respectively, or periphrastically with \texttr{meer}{more} and \texttr{meest}{most} (Gloss~\ref{gloss:superlative}).
Complex adjectival phrases constructed with the aid of prepositions commonly occur immediately after the noun phrase they modify (Gloss~\ref{gloss:adjp}).

\begin{exe}
\ex 
\begin{xlist}
\ex\label{gloss:simple_np}
\gll \textit{een} \textit{mooi} \textit{lied}\\
\abbrv{det.indf} beautiful song\\
\trans{`a beautiful song'}
\ex\label{gloss:adjective_definite_n}
\gll \textit{het} \textit{mooi-e} \textit{lied}\\
\abbrv{det.def.n.sg} beautiful-\abbrv{def} song\\
\trans{`the beautiful song'}
\ex\label{gloss:superlative}
\gll \textit{het} \textit{mooi-ste} \textit{lied}\\
\abbrv{det.def.n.sg} beautiful-\abbrv{sup} song\\
\trans{`the most beautiful song'}
\ex\label{gloss:partitive}
\gll \textit{iets} \textit{mooi-s}\\
something beautiful-\abbrv{ptv}\\
\trans{`something beautiful'}
\ex\label{gloss:adjective_definite_nn}
\gll \textit{de} \textit{klein-e} \textit{vogel}\\
\abbrv{det.def.nn.sg} small-\abbrv{def} bird\\
\trans{`the small bird'}
\ex\label{gloss:diminutive}
\gll \textit{een} \textit{klein} \textit{vogel-tje}\\
\abbrv{det.indf} small bird-\abbrv{dim}\\
\trans{`a small birdie'}
\ex\label{gloss:genitive}
\gll \textit{de} \textit{mooi-e} \textit{lied-eren} \textit{van} \textit{vogel-s}\\
\abbrv{det.def.n.pl} beautiful-\abbrv{def} song-\abbrv{pl} of bird-\abbrv{pl}\\
\trans{`the beautiful songs of birds'}
\ex\label{gloss:adjp}
\gll \textit{geen} \textit{lied} \textit{op} \textit{een} \textit{dod-e} \textit{planeet}\\
\abbrv{det.neg.indf} song on \abbrv{det.indf} dead-\abbrv{indf.nn} planet\\
\trans{`no song on a dead planet'}
\end{xlist}
\end{exe}

\paragraph{Personal Pronouns}
Noun phrases can be substituted by personal pronouns, which in Dutch are morphologically marked for case.
The nominative is used for the subject position, the genitive corresponds to the possessive determiners discussed earlier, and the accussative is used to denote objects.
A dative form is sometimes exceptionally used for the third person plural.
The third person singular has three distinct forms corresponding to grammatical gender; 
depending on the regional variation, nouns must be referred to by their correct gender, or simply by the neuter (in which case the masculine and feminine forms are reserved for animates).
Third person singulars are also interchangeable with the appropriate demonstratives.
Personal pronouns come in two variants: the stressed (emphatic) and the unstressed (standard).

\subsubsection{The Verb}
\paragraph{Conjugation}
In their citation form, verbs match their infinitival versions, regularly consisting of the verbal stem plus \textex{-en}.
Verbal conjugation patterns distinguish between two grammatical tenses, the non-past and the past, and three moods, the indicative, the subjunctive and the imperative, of which the first two are morpohologically conflated.
Each pattern is parameterized by person and number, following the standard West Germanic archetype.
Aspectual flavours, passivization and an explicit future are accessible as productive constructions with modals, the latter being the common culprits of irregular conjugation.

\paragraph{Participles}
Participles exist for both tenses, and have a multitude of uses.
The present participle is formed by affixing \textex{-de}, and is commonly employed as an duration-denoting adjective or adverb, always inflected in the first case, and optionally in the second (Gloss~\ref{gloss:prs_ptcp}).
Rarely, it can be used as a complement to the modal \texttr{zijn}{to be} to produce a kind of present continuous.
Present participles of transitives can attach to the end of their object nouns, appearing as fused compounds.
The past participle is more versatile.
Regular past participles are formed by prefixing \textex{ge-} to the verbal stem (provided it can accept it) and substituting the infinitival suffix for either \textex{-t}, \textex{-d} or nothing, depending on the stem's last letter.
Like the present participle, it can be used as an adjective, denoting now a completed event.
Combined with with the modal \texttr{hebben}{to have} (or exceptionally \textex{zijn} for unaccusatives and verbs of movement), it produces the perfect tense.
Combined with the modals \texttr{worden}{to become} and \texttr{zijn}{to be}, it produces the passive voice and its perfect tense (Gloss~\ref{gloss:pst_ptcp}).

\paragraph{Infinitives}
Infinitival forms commonly occur as the verbal complements of a modal, auxiliary or sensory verb.
Depending on the modal, the preposition \texttr{te}{to}, or the discontinuous \texttr{om ... te}{to}, may either be necessary, optionally admissible or completely disallowed -- if one does manifest, it precedes the infinitive.
The latter can enclose linguistic material, like the infinitive's object or any adverbs modifying it.
An infinitive directly following \textex{aan het} can combine with \textex{zijn} to construct the continuous aspect, in either the present or the past tense.
Infinitives are also often nominalized, the corresponding ``nouns'' being singular neuters.

\paragraph{Separable Verbs} 
The verbal lexicon contains several compound items comprised of a preposition and a verbal stem, usually with a compositional meaning.
The stem is separated from its prepositional prefix when the verb heads a matrix clause, in which case the preposition is moved to the end of the clause -- in all other cases, including the finite form in subordinate clauses, the two remain attached (Gloss~\ref{gloss:svp}).
When inflecting for the perfect, the prefix applies to the stem (i.e. after the preposition, see Gloss~\ref{gloss:pst_ptcp}).

\subsubsection{The Sentence}
By far the most fun aspect of Dutch is its absolutely wild sentential word order.

%\paragraph{Adverbial Phrases}
%Adverbs and adverbial phrases follow 

\paragraph{Main Clauses}
Dutch main clauses emanate a false sense of safety, coming off as SVO at first glance (Gloss~\ref{gloss:simple_smain}).
The truth is far more sinister -- the verb placed there only by exception, abiding by the V2 rule that has it appear second for matrix clauses only.
The effect becomes apparent when employing a preverbal adverb -- in Gloss~\ref{gloss:simple_smain_v2}, both the subject and its predicate complement follow the verb in a VSO pattern.
Participles used for the passive the perfect tense are usually pushed to the end of the matrix clause (Glosses~\ref{gloss:prs_ptcp} and \ref{gloss:pst_ptcp}).

\paragraph{Questions and Imperatives}
When it comes to questions, things look familiar again.
Line in English, wh-questions begin with an interrogative pronoun or adverb (e.g. \texttr{wie}{who}, \texttr{waar}{where}, \texttr{welk}{which}, \texttr{waarom}{why} etc.), which is immediately followed by the conjugated verb (Gloss~\ref{gloss:simple_whq}).
In direct questions without an interrogative, as well as personal positive commands, the verb is placed first (Glosses~\ref{gloss:simple_sv1_q} and \ref{gloss:simple_sv1_imp}).
In negative imperative sentences and impersonal commands, the infinitival is placed last.

\begin{exe}
\ex
\begin{xlist}
\ex\label{gloss:simple_smain}
\gll \textit{Frans} \textit{verkoop-t} \textit{kaas.}\\
Frans sell-\abbrv{prs.3sg} cheese\\
\trans{`Frans sells cheese.'}
\ex\label{gloss:simple_smain_v2}
\gll \textit{Morgen} \textit{verkoop-t} \textit{Frans} \textit{kaas.}\\
tomorrow sell-\abbrv{prs.3sg} Frans cheese\\
\trans{`Frans will sell cheese tomorrow.'}
\ex\label{gloss:simple_sv1_q}
\gll \textit{Verkoop-t} \textit{Frans} \textit{kaas?}\\
sell-\abbrv{prs.3sg} Frans cheese\\
\trans{`Does Frans sell cheese?'}
\ex\label{gloss:simple_whq}
\gll \textit{Wie} \textit{verkoop-t} \textit{kaas?}\\
who sell-\abbrv{prs.3sg} cheese\\
\trans{`Who sells cheese?'}
\ex\label{gloss:simple_sv1_imp}
\gll \textit{Verkoop} \textit{kaas!}\\
sell(\abbrv{imp}) cheese\\
\trans{`Sell cheese!'}
\end{xlist}
\end{exe}

\begin{exe}
\ex
\begin{xlist}
\ex\label{gloss:prs_ptcp}
\gll \textit{De} \textit{spelen-de} \textit{mens} \textit{heeft} \textit{zijn} \textit{huis} \textit{in} \textit{brand} \textit{ge-stoken.}\\
the play-\abbrv{prs.ptcp} man has his house in fire \abbrv{pst.ptcp}-put\\
\trans{`The playing man has set his house on fire.'}
\ex\label{gloss:svp}
\gll \textit{Het} \textit{huis} \textit{brand-t} \textit{af}.\\
the {house} {burn}-\abbrv{prs.3sg} down\\
\trans{`The house burns down.'}
\ex\label{gloss:pst_ptcp}
\gll \textit{Het} \textit{huis} \textit{is} \textit{af\textlangle ge\textrangle brand.}\\
the {house} {is} \textlangle\abbrv{pst.ptcp}\textrangle{burn}\\
\trans{`The house has burnt down.'}
\end{xlist}
\end{exe}

\paragraph{Subordinate Clauses}
Subordinate clauses is where things really get interesting.
Unaffected by the V2 rule, the worder order turns out to be SOV; this affects indirect questions, verbal complements and relative clauses alike.
Indirect questions are straightforward -- modulo the word order permutation, they match their direct counterparts (Gloss~\ref{gloss:embedded_q}).
Infinitives in verbal complement position are likewise just pushed to the end of their clause.
Relative clauses are instigated by relative adverbs and pronouns.
Interestingly, the language does not make an overt distinction between an object- and a subject- relative pronoun; combined with the SOV word order, and the absence of case markings, the effect is that the two relative clause types end up having the exact same surface form when the grammatical gender of the antecedent noun and the non-gap embedded argument are the same (contrast the two examples of Gloss~\ref{gloss:rc_ambiguity}).

\begin{exe}
\ex\label{gloss:embedded_q}
\gll \textit{Weet} \textit{je} \textit{wie} \textit{kaas} \textit{verkoopt?}\\
know you who cheese sell-\abbrv{prs.3sg}\\
\trans{`Do you know who sells cheese?'}
\ex\label{gloss:simple_vc}
\gll \textit{Frans} \textit{wil} \textit{koopman} \textit{worden}\\
Frans wants merchant be(\abbrv{inf})\\
\trans{`Frans wants to be a merchant'}
\ex\label{gloss:rc_ambiguity}
\begin{xlist}
\ex
\gll \textit{het} \textit{huis} \textit{dat} \textit{vuur} \textit{opslok-t}\\
the house(\abbrv{n}) that(\abbrv{n}) fire(\abbrv{n}) consume-\abbrv{prs.3sg}\\
\trans{
\begin{enumerate}[topsep=0pt]
	\item (\#) `the house that consumes fire'
	\item `the house that fire consumes'
\end{enumerate}}
\ex
\gll \textit{het} \textit{huis} \textit{dat} \textit{de} \textit{man} \textit{in} \textit{brand} \textit{steekt}\\
the house(\abbrv{n}) that(\abbrv{n}) the man(\abbrv{nn}) in fire puts\\
\trans{`the house that the man sets on fire'}
\end{xlist}
\end{exe}

The SOV order means that the chaining of verbs requiring non-finite complements inadvertently leads to verb clusters, i.e. collections of two or more verbs situated within the dependent clause and adjacent to one another.
Verb clusters are marked by their inability to accommodate non-verbal material, and may follow a number of different word orders, which don't necessarily abide by the order of selectional dominance (Gloss~\ref{gloss:rusten_zal}).
The question of which factors influence the grammaticality of word order variations  is a hot potato and a topic of active research for decades -- to make matters worse, these factors tend to differ between regional variations of the language%
	\footnote{As a fun trivia,  out of the 6 possible orderings of 3-verb clusters, 4 to 5 were found admissible by Dutch speakers depending on the construction~\cite{3vc}.}.
What follows are some simplified common observations -- the interested reader should find the thesis of~\citet{augustinus2015complement} a good entry point.

Past participles used in the formation of the perfect or passive, for starters, may occur either to the left or the right of tense the auxiliaries \textex{hebben} and \textex{zijn}, leading to either a German- or English- like construction.
This gets complicated by the so-called IPP (\textit{Infinitivus Pro Participio}) effect, where a participle that selects for an infinitive changes to an infinitive itself, creating a cluster in the process -- once more, whether this substitution is mandatory, optional or altogether impossible is lexically decided (Gloss~\ref{gloss:ipp}).
Next, the infinitival head of a dependent clause may be forced to occur directly after the verb dominating it, if the latter belongs to a closed set of so-called \textit{raising} verbs (i.e. the raiser is infixed between the infinitive to the right, and the infinitive's object to the left).
These include modals like \texttr{willen}{want}, \texttr{zullen}{will} and \texttr{moeten}{must}, perception verbs like \texttr{horen}{hear} and \texttr{zien}{see}, and some nondescript verbs like \texttr{doen}{to do} and \texttr{laten}{to let}.
As before, they can be subcategorized as obligatory raisers and optional raisers.
Other verbs like \texttr{verplichten}{forbid} select not a bare infinitive but a \textex{te}-marked infinitival phrase, which they leave intact at the end of the clause in a phenomenon known as \textit{extraposition}.
The twist is that the intersection of extraposition verbs and raising verbs is non-empty (e.g. \texttr{proberen}{to try} can behave as either, see Gloss~\ref{gloss:vr_vs_xpos}).

\begin{exe}
\ex\label{gloss:rusten_zal}
\begin{xlist}
\ex
\gll \textit{waar} \textit{ik} \textit{naamloos} \textit{rusten} \textit{zal}\\
where I nameless rest(\abbrv{inf}) will\\
\ex
\gll \textit{waar} \textit{ik} \textit{naamloos} \textit{zal} \textit{rusten}\\
where I nameless will rest(\abbrv{inf})\\
\trans{`where I will rest nameless'}
\ex[*]{\textit{waar} \textit{ik} \textit{zal} \textit{naamloos} \textit{rusten}}
\end{xlist}
\ex\label{gloss:ipp}
\begin{xlist}
\ex 
\gll \textit{Ik} \textit{heb} \textit{de} \textit{eend} \textit{ge-zien.}\\
I have the duck \abbrv{pst.ptcp}-see\\
\trans{`I have seen the duck.'}
\ex
\gll \textit{Ik} \textit{heb} \textit{de} \textit{eend} \textit{zien} \textit{vliegen.}\\
I have the duck see(\abbrv{inf}) fly(\abbrv{inf})\\
\trans{`I have seen the duck fly.'}
\ex[*]{\textit{Ik} \textit{heb} \textit{de} \textit{eend} \textit{gezien} \textit{vliegen.}}
\ex
\gll \textit{Ik} \textit{heb} \textit{de} \textit{eend} \textit{een} \textit{vis} \textit{zien} \textit{eten.}\\
I have the duck a fish see(\abbrv{inf}) eat(\abbrv{inf})\\
\trans{`I have seen the duck eat a fish.'}
\ex[*]{\textit{Ik} \textit{heb} \textit{de} \textit{eend} \textit{zien} \textit{een} \textit{vis} \textit{eten.}}
\end{xlist}
\ex\label{gloss:vr_vs_xpos}
\begin{xlist}
\ex
\gll \textit{Ik} \textit{denk} \textit{dat} \textit{hij} \textit{probeert} \textit{iets} \textit{te} \textit{zeggen}.\\
I think that he tries something to say\\
\ex
\gll \textit{Ik} \textit{denk} \textit{dat} \textit{hij} \textit{iets} \textit{probeert} \textit{te} \textit{zeggen}.\\
I think that he something tries say to say\\
\trans{`I think that he is trying to say something'}
\end{xlist}
\end{exe}

Typology aside, Dutch verb clusters have been a favorite topic of debate for formal grammarians for a while now, since their construction requires expressive capacity beyond what a context-free grammar can offer, and thus brinking an end to any delusion that human languages are context-free%
	\footnote{Or, depending on the reader, that Dutch is a human language.}%
~\cite{huybregts1984weak,shieber1985evidence}.



% ADVERBS?
% PRONOMINAL
% CONJUNCTION




%\begin{figure}
%	\begin{tikzpicture}
%	\begin{semilogxaxis}[
%	    title={Type Assignments/Occurrence Counts},
%	    xlabel={Occurrence Count},
%	    ylabel={Proportion of Type Assignments},
%	    xmin=1, xmax=140000,
%	    ymin=-0, ymax=1,
%	    xtick={1,10,100,1000,10000, 100000},
%	    legend pos=north west,
%	    ymajorgrids=true,
%		minor y tick num=1,
%	    yminorgrids=true,
%	    xmajorgrids=true,
%	    xminorgrids=true,
%	    axis line style={draw=none},
%	    tick style={draw=none}
%	]
%	
%	% types
%	\addplot[black, const plot, dashdotdotted, thick]
%		table[
%	    mark=none,
%	    x index=0,
%	    y index=1,
%	    col sep=comma,
%	    ] {data/proportion_of_types_covered.dat};
%	
%	% assignments
%	\addplot[black, const plot, dashed, thick]
%		table[
%	    mark=none,
%	    x index=0,
%	    y index=1,
%	    col sep=comma,
%	    ] {data/proportion_of_assignments_covered.dat};
%	
%	% sentences
%	\addplot[black, const plot]
%		table[
%	    mark=none,
%	    x index=0,
%	    y index=1,
%	    col sep=comma,
%	    ] {data/proportion_of_sentences_covered.dat};
%	    
%	\end{semilogxaxis}
%	\end{tikzpicture}
%	\label{figure:plot:assignment_occurrence_ecdf}
%\end{figure}

%The SOV word order of subordinate clauses can at times be a major source of headaches.
%For starters, Dutc relative pronouns
%For starts, when the object position is occupied by a verbal complement, 
% not a real language


% motivate non-directionality

% CITE AUGUSTINUS!!


\nocite{augustinus2015complement}

\bibliographystyle{abbrvnat}
\bibliography{bibliography}
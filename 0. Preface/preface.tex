\chapter*{Preface}

Greetings, reader. Out of coincidence, or some weird turn of events, I have written this dissertation to-be and you have stumbled upon it.
Introductions would normally be in order, but since this communication channel is asynchronous and unidirectional I will be doing double duty for the both of us.

So, let's start with you. 
A few scenarios are plausible as to why you are browsing these pages. 
Most likely you are an acquaintance of mine, either social – in which case you are wondering what it is I spent 5 years in Utrecht for, or academic – which means you are probably trying to figure out whether I am worthy of the title of doctor. 
In the latter case, I hope you won’t be disappointed (especially so if you happen to be a member of the examination committee, for both our sakes).
Or, perhaps, you are lazilly scrolling through the opening pages to evaluate my fitness for a potential job in an organization you are representing? 
If so, you should definitely go for me – unless this happens to be any of the big five%
	\footnote{Preemptive apologies for not making sense to readers after the imminent tech collapse.}%
, in which case: shoo, and shame on you, future me.
Otherwise, could it even be that you are actually interested in the subject matter of this thesis? 
That would be a shock; pleasant mostly, but also slightly alarming: I feel a bit conscious knowing that you might be putting my words under a critical lens – I’ll do my best not to fail your expectations. 
In the unlikely event that you do not fall in any of the above categories, excuse my lack of foresight and know that you are still very welcome, and I am happy to have you around. 
In the more likely event that nobody ever reads this (far), let this transmission be forever lost to the void.

But enough with you, what about me? 
At the time of writing, I am in my early thirties and I call myself Kokos. 
I had the enormous luck of crossing paths with my supervisor, Michael, five years ago, during the first weeks of my graduate studies in Utrecht. 
The repercussions of this encounter were (and still are) unforeseeable. 
Coming from an engineering background that basked in practicality, with an obsessive repulsion to anything formal, his course offered me a glimpse of a whole new world. 
I got to see that proofs are not irrelevant bureaucracies to avoid, but objects of interest in themselves, hidden in plain sight from the working hacker under common programming patterns. 
If this naive revelation came as shock, you can imagine my almost mystical awe when I was shown how proof \& type theories also offer suitable tools and vocabulary for the analysis of human languages. 
Despite my prior ignorance, the “holy trinity” between constructive logics, programming languages and natural languages has been (with its ups and downs) at the forefronts of theoretical research for well over a century. 
This dissertation aims to be my tiny contribution to this line of work, conducted from the angle of a late convert, a theory-conscious hacker. 

If all this sounds enticing and you plan on sticking around, at least for a bit longer, I think it would be beneficial if we set down the terms and conditions of what is to follow. 
It is no secret that dissertations are often boring to read, and it can be easy to lose track of context in seemingly unending walls of text. 
Striking a balance between being pedantic and making too many assumptions on background knowledge is no easy task: the only way to spare you unecessary headaches requires a mutual contract. 
On my part, I will try to clearly communicate my intentions, both about the thesis in full, and its parts in isolation: the idea is to make this manuscript as self-contained as possible, but without nitpicking on details or taking detours unnecessary for the presentation of the few novelties I have to contribute.
Of you, I ask to remain conscious of what you are reading and aware of my own biases and limitations. 
The absence of feedback means that I can only model you in my imagination; 
I will inadvertently skip things that to me seem self-evident, and rant at length about others that you take for granted.
So, feel free to skip ahead when something reads trivial, and do not judge too harshly when you encounter an explanation you find insufficient.

\subsection*{What this thesis is about}
\todo \\
contributions \\
sectioning \\ 


%\subsection*{What this is all about}
%The overarching theme of this study are typelogical grammars, with a focus on their applicability in realistic scenarios at scale.
%The 
%
%In Chapter~\ref{chapter:Introduction}, we set up the historical background that serves as the implicit basis of many of the concepts that will concern us in what is to come. 
%It contains a brief introduction to intuitionistic logics and their computational counterparts, known to the initiates as $\lambda$-calculi.
%\subsection*{How to read it}
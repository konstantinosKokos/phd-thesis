{
\renewcommand{\thechapter}{\arabic{chapter}}
\setcounter{chapter}{-1}
\renewcommand{\bibname}{Papers this dissertation is based on}
\chapter{Preface}
\label{chapter:preface}

Greetings, reader. Out of coincidence or some weird turn of events, I have written this dissertation to-be and you have stumbled upon it.
Introductions would normally be in order, but since this communication channel is asynchronous and unidirectional I will be doing double duty for the both of us.

So, let's start with you. 
A few scenarios are plausible as to why you are browsing these pages. 
The most likely scenario is that you are an acquaintance of mine; either social or academic.
In the first case, you are probably wondering what it is I spent 5 years in Utrecht for.
In the second case, you are probably trying to figure out whether I am worthy of the title of doctor; if so, I hope you won’t be disappointed (especially if you happen to be a member of the examination committee, for both our sakes).
Another plausible scenario is that you're just lazilly scrolling through the opening pages contempating whether I'd be a good fit for some organization you are representing?
If so, you should definitely go for me.
Exceptionally, if this happens to be some big corp reptile den, scram -- and shame on you, future me.
Otherwise, could it even be that you are \textit{actually} interested in the subject matter of this thesis? 
Wooo, exciting!
But also slightly alarming.
I feel a bit conscious knowing that you might be putting my words under a critical lens – I’ll do my best not to fail your expectations. 
In the wildcard scenario where you do not fall in any of the above categories, excuse my lack of foresight and know that you are still very welcome, and I am happy to have you around.%
\footnote{More realistically, if noone ever reads this (far), let this transmission be forever lost to the void.}

But enough with you, what about me? 
At the time of writing, I am in my early thirties and I call myself Kokos. 
I had the enormous luck of crossing paths with my supervisor, Michael, five years ago, during the first weeks of my graduate studies in Utrecht. 
The repercussions of this encounter were (and still are) unforeseeable. 
Coming from an engineering background that basked in practicality, with an obsessive repulsion to anything formal, his course offered me a glimpse of a whole new world. 
I got to see that proofs are not irrelevant bureaucracies to avoid, but objects of interest in themselves, hidden in plain sight from the working hacker under common programming patterns. 
If this naive revelation came as shock, you can imagine my almost mystical awe when I was shown how proof \& type theories also offer suitable tools and vocabulary for the analysis of human languages. 
Despite my prior ignorance, the “holy trinity” between constructive logics, programming languages and natural languages has been (with its ups and downs) at the forefronts of theoretical research for well over a century. 
This dissertation aims to be my tiny contribution to this line of work, conducted from the angle of a late convert, a theory-conscious hacker. 

If all this sounds enticing and you plan on sticking around, at least for a bit longer, I think it would be beneficial if we set down the terms and conditions of what is to follow. 
It is no secret that dissertations are often boring to read, and it can be easy to lose track of context in seemingly unending walls of text. 
Striking a balance between being pedantic and making too many assumptions on background knowledge is no easy task: the only way to spare you unecessary headaches requires a mutual contract. 
On my part, I will try to clearly communicate my intentions, both about the thesis in full, and its parts in isolation: the idea is to make this manuscript as self-contained as possible, but without nitpicking on details or taking detours unnecessary for the presentation of the few novelties I have to contribute.
Of you, I ask to remain conscious of what you are reading and aware of my own biases and limitations. 
The absence of feedback means that my mental model of you is a purely artificial construct of my imagination.
I will inadvertently skip things that to me seem self-evident, and rant at length about others that you take for granted.
So feel free to skip ahead when something reads trivial, and do not judge too harshly when you encounter an explanation you find insufficient.

\subsection*{What this thesis is about}
The quote below was received almost verbatim as a review.
Mean spirited as it may be, it provides an adequate high-level summary of this thesis' contents: 
\begin{quote}
[The thesis] starts with Lambek Calculus, some how uses dependency labels in some of its semantic types, provides a parsing algorithm for it; there are neural networks and vectors used and some accuracy results provided, but I am still unsure about the contributions [of the thesis] and their relevance.
\begin{flushright} Unknown reviewer, 2019.\end{flushright}
\end{quote}

Thanks to this fellow scientist's earnest reviewing work, all I have to elucidate here is the thesis' breakdown and its contributions. Starting from the former:
\begin{enumerate}[labelindent=2pt, itemindent=30pt, labelsep=5pt, widest=Chapter III,align=right,itemsep=5pt]
\item[\textbf{Chapter~\ref{chapter:preface}}] greets the reader and provides a chapter summary, while also setting the tonal precedents for what is to follow.
\item[\textbf{Chapter~\ref{chapter:Introduction}}] makes an attempt at a painless introduction to simple type theory and its substructural variants, with an emphasis on linguistics and type-logical grammars.
We start in Section~\ref{section:simple_type_theory} with a crash course on simple type theory, followed by a transition to a linear types in Section~\ref{section:linear_type_theory}.
Further into substructural territory, we take a look at Lambek calculi in Section~\ref{section:lambek_calculi}, where more than just linear, our types become immovable and bracket-bound.
To regain some of the expressivity lost in the passage, we call to the aid of modalities in Section~\ref{section:modalities}.
We deploy the above as categorial grammars, type-driven frameworks for reasoning about natural language syntax and semantics in Section~\ref{section:linguistics}, with a focus on type-logical grammars and abstract categorial grammars.
\item[\textbf{Chapter~\ref{chapter:chapter_2}}] offers a non-standard usecase for the structural control modalities of the multimodal Lambek family $\logic{(N)L(P)}_{\diamond, \bx}$.
We begin in Section~\ref{section:phrase_vs_dependecy} with a face-off between the two strands of grammar flavors that have dominated computational linguistics in the past decades, and see how they compare to categorial grammars.
Unsatisfied by the comparison, we move on to Section~\ref{section:modalities_for_dependency}, where we appropriate the modalities resorted to earlier, repurposing them now as dependency domain demarcators.
\item[\textbf{Chapter~\ref{chapter:chapter_3}}] instantiates the type system as a derivational semantics logic (or abstract syntax logic, depending on which side of the dividing line you stand at), and aligns it with real-world corpus data. 
We set the stage in Section~\ref{section:preliminaries} with a backstory intended to motivate the design choices behind the logic, and a description of the corpus used.
We then proceed to describe the proof extraction process and its evaluation in Section~\ref{section:aethel}.
\item[\textbf{Chapter~\ref{chapter:chapter_4}}] makes for a change of scenery, offering a collection of insights on the neural parsing of substructural grammar logics.
We first paint a picture of the archetypical categorial grammar parser in Section~\ref{section:parse}, pinpointing the tension points between abstract theory and applied practice.
We then trace along the history of supertagging in Section~\ref{section:supertagging}, going from its (not-so-ancient) origins all the way to tomorrow.
We motivate the abolition of the predefined lexicon as a natural step in its evolutionary progress, and provide two convincing implementations to that end.
Finally, in Section~\ref{section:npn} we propose a neural operationalization for the proof nets of linear logic, and show how this yields a brand new paradigm for parsing substructural grammar logics.
\item[\textbf{Chapter~\ref{chapter:chapter_5}}] provides some concluding remarks and waves the reader goodbye.
\end{enumerate}

\paragraph{Contributions}
Contributions produced and presented in this thesis, organized in bullet points for your convenience and reading pleasure, include:
\begin{itemize}
\item a \textbf{type-driven} model of compositional syntax that simultaneously captures \textbf{dependency-} and \textbf{function-argument- structures}; not a world first, but a close second, by 30 years
\item a big, open-source, \textbf{well-typed dataset} of proof-derivations for written Dutch
\item the \textbf{first supertagger} to correctly \textbf{construct novel type assignments}, operating without a fixed lexicon
\item the current \textbf{state of the art} supertagger that outperforms accuracy benchmarks \textbf{across grammar frameworks}, without foregoing the ability to predict rare and unseen assignments
\item a \textbf{neural operationalization} of linear logic \textbf{proof nets} into a massively parallel, differentiable and performant proof search engine
\item several attempts at dry humor, to varying degrees of success; in the words of my beloved: ``this thesis is a joke''
\end{itemize}


\paragraph{Publications}
Chapter~\ref{chapter:chapter_2} is a novel, extended collage of work taking secondary role in \citet{kogkalidis-etal-2020-aethel} and \citet{rouss}.
Chapter~\ref{chapter:chapter_3} is an extended version of \citet{kogkalidis-etal-2020-aethel}.
Chapter~\ref{chapter:chapter_4} is based on \citet{kogkalidis-etal-2019-constructive,kogkalidis-etal-2020-neural} and \citet[preprint]{kogkalidis2022geometryaware}.
The whole manuscript is a bigger, better, faster, stronger (or so I'd like to think) version of early work delivered as part of my master's thesis~\cite{https://doi.org/10.48550/arxiv.1909.02955}.


\bibliographystyle{abbrvnat}
\bibliography{bibliography}
}